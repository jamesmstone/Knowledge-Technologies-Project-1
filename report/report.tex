\documentclass[a4paper]{article}
%\usepackage[margin=0.75in]{geometry} % margins
\usepackage[english]{babel}
\usepackage[utf8]{inputenc}
\usepackage{amsmath}
\usepackage{graphicx}
\usepackage[colorinlistoftodos]{todonotes}
\usepackage{float} % Image float [H] option
\usepackage{subfig} % Figures in figures ( I think)

\usepackage{hyperref} % \url

% Default Font
\usepackage[default]{lato}
\usepackage[T1]{fontenc}
\renewcommand{\mddefault}{l}% switch default weight to light

% Paragraphs
\setlength{\parskip}{\baselineskip}	% add space between paragraphs
\setlength{\parindent}{0pt}			% No paragraph indent

% Declare first level dot point as a -
\def\labelitemi{--}

% Smart quotes
\usepackage [autostyle, english = american]{csquotes}
\MakeOuterQuote{"}

\title{Knowledge Technologies - Project 1: Misspelled Location Names}

\author{James Stone - 761353}

\date{August 2016}

\begin{document}
\maketitle

\section{Description}
%1. A basic description of the problem and the data set;
The task was to filter a list of tweets to ones including locations.

The data was sourced from two files, One containing locations in the US and one containing tweets.
Unfortunately, both file needed some pre-processing to make them usable. (see Method Overview)
\section{Method Overview}
%%2. An overview of your approximate matching methods. You can assume that the reader is familiar with the methods discussed in this subject, and instead focus on how they are applied to this task,

I choose to treat the sample as a monolithinc entity and use Neighbourhood Search.
I choose to write a Bash script: \texttt{neighbourhoodSearch.sh}.

\subsection{Pre-processing}
Before the neighbourhood search is run I had to pre-process the data. To do this I performed the following operations:

\subsubsection{Tweets file}
I removed everything that wasn't directly related to the \textit{"raw"} tweets. This meant removing:
\begin{itemize}
  \item User ID
  \item Tweet IDs
  \item Timestamps
\end{itemize}

\subsubsection{Locations file:}
I also performed the following operations on the Locations file:
\begin{description}
\item[Removed words that contain numbers] A measure to prevent lots of false positives (tweets containing numbers not involving a location), however it increases the number of false negatives (tweets about locations that have a number in their name).
\item[Removed punctuation] As a measure to help limit the number of duplicate searches, place name with punctuation will still get picked up due to th
\item[Replaced "-" and "/" with " "]
\item[Removed common small words] such as: '\textit{of}', '\textit{and}', '\textit{a}', '\textit{an}' and '\textit{the}'. This was done as these words aren't unique to place names. Additonally, they are more common in other parts of language and therefore would lead to more false positives than positives.
\item[Replaced spaces with newlines] (Only after initial run, see section below on Multiple word places).
\item[Change case] in preparation for removing duplicates, I converted all the text to it's lowercase version.
\item[Removed lines that contain numbers written as words] e.g. "\textit{hundred}" and "\textit{thousand}". This was done for the same reason I removed lines numbered lines.
\item[Remove duplicates] No uses processing the same result multiple times.
\item[Trim location length to 20 characters] As the tool I'm using to perform the neighbourhood search, \texttt{agrep}, only allows search patterns of a maximum length. Additionally places that are greater than 20 characters are more likely not present in an 140 character tweet. Further, the longer the search string the longer it takes to perform the neighbourhood search.
\end{description}

% including some indication of how you dealt with location names of more than one word (if necessary);
\subsection{Multiple word places}
Another difficulty, was places that were more than one word long: eg “\textit{New York}” these posed a problem as other places were unnesasacirly longer than one word. For example:Zion churches there was no consistency:
\begin{itemize}
\item Zion Congregational Church of God in Christ
\item Zion Dominion Church of God
\item Zion Hill Baptist Church of East Detroit
\item Zion Hill Church of God in Christ
\item Zion Light Church of Christ
\item Zion Lutheran Church of Augsburg
\end{itemize}

Due to the size restrictions of \texttt{agrep} and time constrains of searching, it wasn't feasible to search for approximate matches of the whole string "\textit{Zion United Church of Christ Cemetery}" particularly as the longer the string the less likely we are to match the entire phrase, even with an approximate search. Additionally you could miss cases where it has been written in an active voice over a passive voice of vice versa. For example the previous location could have easily been written as the "Christ Church Cemetery of the Zion United Church". As such I decided to split the words on spaces. However, that brings its own issues. Now the data set was matching tweets that contained smaller common words. For example it was matching all tweets that contain  "\textit{of}" from "\textit{Church of}" or "\textit{New}" from "\textit{New York}".

The dataset also contained many locations joined together via hyphens, some of these were joined for genuine reasons such as "" others however were joined as together they

\section{Effectiveness}
%3. A discussion of the effectiveness of the approximate matching method(s) you chose,
% including a formal evaluation, and some examples to illustrate where the method(s) was/were effective/ineffective;
Unffortunately, the location dataset was all over the place. It contained many duplicates (not a problem as relatively easy and time efficient to filter out).
%  it also contain a collection of places, such as a series of numbers written out as words  eg "onethousdandtwohundred". It also contained a small collection of urls.

In the output data I noticed where lots of false positives due to the splitting of places into individual words: (see section on Multiple word places, as to why this was done). Examples of this can be seen below:
\begin{itemize}
\item Location: \texttt{valley}:  \textit{on the "\textbf{valley} girls" episodes of gossip girl. i love this episode!!}
\item Location: \texttt{movie}:   \textit{"(500) Days Of Summer" is basically "HIMYM: The \textbf{Movie}": playful about timelines and structure, w/a protagonist who wants too much too fast.}
\item Location: \texttt{green}:   \textit{RT @MarieLancup RT @EcoChic: RT @DawnSandomeno: Top 5 Tips for "\textbf{Green}" Entertaining http://www.partybluprintsbl...}
\end{itemize}

%James: Talk about precision
This task can only be analyised in a qualatative manner, as I have no way of knowing definetively if a tweet contains a location. Therefore, it is not possible to calculate the effectiveness in terms of preicision or accuracy.

\section{Conclusion}
%4. Some conclusions about the problem of searching for misspelled location names within a collection of tweets

\end{document}

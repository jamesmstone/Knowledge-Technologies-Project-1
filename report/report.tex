\documentclass[a4paper]{article}

\usepackage[english]{babel}
%\usepackage[utf8]{inputenc}
\usepackage{amsmath}
\usepackage{graphicx}

% \usepackage{natbib}
% \usepackage{fixltx2e}
% \usepackage[version=3]{mhchem}
% \usepackage{float}

\usepackage{titlesec}
\usepackage{titling}
\usepackage{fontspec}
\usepackage{authoraftertitle}

% \setmainfont{Lato}
% \setsansfont{Trebuchet MS}
% \setmonofont{Inconsolata}
% Specify different font for section headings
% \newfontfamily\headingfont[]{Georgia}
\titleformat*{\section}{\LARGE\headingfont}
\titleformat*{\subsection}{\Large\headingfont}
\titleformat*{\subsubsection}{\large\headingfont}
%\renewcommand{\maketitlehooka}{\headingfont}

\bibliographystyle{agsm}

\title{Knowledge Technologies - Project 1: Misspelled Location Names}

\author{James Stone - 761353}

\date{August 2016}

\begin{document}
\maketitle

\section{Description}
%1. A basic description of the problem and the data set;
\section{Method Overview}
%2. An overview of your approximate matching methods. You can assume that the reader is familiar with the methods discussed in this subject, and instead focus on how they are applied to this task,
I choose to treat the sample as a monolithinc entity and use Neighbourhood Search.
I choose to write a Perl script:  `neighbourhoodReplace.pl`
% including some indication of how you dealt with location names of more than one word (if necessary);
\section{Effectiveness}
%3. A discussion of the effectiveness of the approximate matching method(s) you chose,
% including a formal evaluation, and some examples to illustrate where the method(s) was/were effective/ineffective;
\section{Conclusion}
%4. Some conclusions about the problem of searching for misspelled location names within a collection of tweets

\end{document}
